% 学位论文的正文应以《绪论》作为第一章
\chapter{绪论}\label{chapter_introduction}

\section{研究背景}

数值计算\cite{rburden81:numerical}是一项在各学科各专业中被广泛运用的一项技术。近些年来,随着计算机科学与技术的高速发展,计算机已成为日常工作、生活中不可或缺的工具,在这种情况下,数值计算与计算机的联系变得更为密切,其应用也日益广泛。与计算机科学的其他方向不同,数值计算中的问题和方法有其鲜明的特点,它不仅仅处理普通的离散的物理量,而且还需要处理连续的物理量,例如时间、距离、温度、电压等等。同时,数值计算涉及的问题很多都是连续的数学问题,例如求导数、求积分或非线性方程等等,理论上不能通过有限步计算出准确的结果,计算过程往往需要近似。数值计算程序需要通过有限步的迭代得到一个充分接近准确解的近似解。然而,由于计算机不能够准确的表示所有的实数,数值计算程序的计算过程中几乎每一步都存在误差,因此,通过一定的技术手段来保证数值计算程序的正确性相当的重要。

迄今为止,数值计算程序已经被广泛地使用在人们生活和工作中的诸多复杂系统之中,其中包括航空航天、国防、交通运输、核电能源和医疗卫生等诸多安全关键系统。之所以称它们为安全关键系统,是因为它们一旦失效将会导致巨大的经济损失,甚至危害人们的生命安全。然而大多数的数值计算程序都是由浮点类型编写,由浮点类型计算产生的舍入误差而导致的程序缺陷非常的难以检测以及调试。在历史上有与浮点数的误差积累要引起的严重的软件安全事故也是数不胜数,例如1996年6月4日,阿丽亚娜5号运载火箭在发射后37秒被迫自行引\cite{lions1996ariane},其原因是因为火箭控制系统中控制火箭水平加速度的一个64位浮点数在强制转换为16位整型数值时发生了溢出,导致程序崩溃后处理器发生数字溢流,将感测角度的垂直读值错误的代入到水平值做运算,导致火箭在高速下进行90度水平滚转而崩解,触发自毁装置的启动。类似阿丽亚娜的控制系统,许多的软件系统使用浮点类型程序来编写实数类型算法的代码,这样的算法对于实数类型也许是正确完备的,然而由于浮点数计算误差以及计算过程中的误差积累,按照这样的方式编写的数值计算程序难以保证其正确性。

除了浮点类型的数值计算程序的正确性问题以外,数值计算程序还面临着执行效率以及内存占用的问题。当浮点类型的数值计算程序不能够满足人们对精度的需求时,开发人员会选择使用更高精度的数值类型来实现其程序。这样固然可以使得程序的计算结果更为准确,然而提高程序精度的代价便是牺牲了程序的执行效率。相关工作表明,同样的计算过程的浮点精度实现以及高精度实现,其执行效率可以相差成百上千倍,这样的情况在某些计算资源有限的场景下是不可容忍的,例如移动设备、嵌入式设备等。

\section{研究现状}
为解决数值计算程序效率以及精度的问题,学术界已经开展了相关的研究工作,其中与我们工作比较相关的工作主要包括程序优化技术以及程序验证技术,下面我们便从这两个方面阐述一下当前相关工作的研究进展。
\subsection{程序优化}

程序优化技术\cite{Sedgewick:1984:ALG:42466}主要是通过对程序进行修改、调整或者重新编写以提高程序某方面的性能,例如使得程序的内存占用更少,程序执行效率更高,程序更加易读易维护等等。针对数值计算程序的优化工作,其优化目标便主要集中在如何提升程序的准确性以及如何提高程序的执行效率这两点上。常用的C语言的编译器GCC\cite{Stallman:2009:UGC:1593499}便自带了一个-ffast-math的编译选项,可以在编译时对程序中的数值计算过程进行优化,提高了程序的运行效率,然而其并不能够保证优化前后的计算结果保持一致,也无法保证优化后的计算结果更为准确。

Eric等\cite{Schkufza:2014:SOF:2666356.2594302}将浮点类型的数值计算程序的优化过程转化成了一个随机的搜索过程,并实现了一个浮点类型程序的自动优化工具STOKE,该工具能够提升程序的执行效率并且其结果的准确性也有一定的保证。Precimonious等\cite{6877460}也开展了一项针对浮点精度的数值计算程序的优化工作,其主要技术手段是通过降低数值计算程序中某些中间变量的精度来达到提升程序执行效率以及减少程序内存占用的效果。

\subsection{程序验证}

程序验证\cite{sanghavi2010formal}主要是通过一定的技术手段来保证数值程序的正确性,包括一些静态分析方法\cite{louridas2006static},动态分析方法\cite{chow2008decoupling},数学证明、约束求解\cite{hess1997lincs}等等。

某些特定的数值计算程序已经能够被证明是计算正确的了,例如在Sylvie\cite{boldo:inria-00171497}验证了计算判别式$b*b-a*c$的数值计算程序的正确性,以及一些简单的计算偏微分方程组的数值计算程序的正确性也得到了验证\cite{boldo:hal-00649240}。上述工作针对了某些特定的数值计算程序,并且是人工的去完成了整个验证过程,现在已经有一些自动化的数值计算程序的验证方法被提出。Rosa\cite{Darulova:2014:SCR:2578855.2535874}提出了一种结合了SMT约束求解器、区间分析以及仿射分析技术的对数值计算程序误差范围的自动化验证工具,FPTaylor\cite{10.1007/978-3-319-19249-9_33}工具使用了泰勒展开公式以及全局优化方法来克服在程序验证过程中普遍存在的对误差的高估的问题。

除了上述工作以外,人们还实现了一些程序分析工具来分析数值计算程序的准确性。例如Fluctuat\cite{Goubault:2011:SAF:1946284.1946301}使用了抽象释义的技术来静态地分析浮点类型的数值计算程序中误差的传播过程。FPDebug\cite{Benz:2012:DPA:2345156.2254118}则使用了一种动态执行的方法,其在数值计算程序的执行过程中将每个变量对应了一个更高精度的变量,以分析程序执行过程中的误差情况。

\section{本文工作}

为了解决前文提到的数值计算程序的正确性与效率的问题,本文提出了一种数值程序的自动优化方法,该方法可以将任意精度类型实现的计算程序自动的转换为高效并且计算稳定的浮点类型的数值计算程序。软件开发人员在开发数值程序时只需要按照在数学上计算该问题的公式以任意精度类型来编写代码,这样的直接按照数学公式来编写的代码是非常清晰的,易于理解也易于维护,紧接着,我们的优化方法可以将这样的代码自动的转换成为高效并且计算稳定的浮点类型的数值程序。在对程序进行转换的过程中,我们并不是对程序中的部分代码片段进行转换,而是将程序的计算过程作为一个整体提取出来,然后对这样一个整体的计算过程进行优化。

在通常情况下,开发一个正确且高效的数值程序需要丰富的数值计算的相关知识,而本文的工作实际上使得软件开发人员在不具备这种专业知识的情况下也能够开发出正确性与效率都得到保障的数值程序。软件开发人员不需要将精力集中在如何处理浮点数的舍入误差,也不必考虑浮点数计算过程中需要避免的一些容易产生较大误差的操作,只需要从数学的角度思考问题,将程序中所有变量的类型看作是实数类型,不必考虑其精度。这样一来,不仅大大提升了数值程序的开发效率,还使得开发出来的数值程序非常的符合人们的直觉,也易于维护。

要完成这样的数值程序的自动优化工作,我们首先会对程序的计算过程进行一个整体的提取工作,这个提取过程主要通过符号执行\cite{Cadar:2013:SES:2408776.2408795}的技术来完成,然后我们寻找一个新的计算过程来替换原有的计算过程,该计算过程需要满足以下两个条件:首先新的计算过程必须与原程序的计算过程在数学意义上是等价的,其次新的计算过程的浮点精度实现在其输入域上必须是计算稳定的。

如何寻找满足这样的条件的计算过程是本文工作的主要难点。在本文中,我们提出了一种基于随机代数变换的转换方法来解决这一问题。数值计算专家们定义了一系列数值计算的等价转换规则,我们在原数值程序的计算过程上递归地应用这些等价的转换规则,如此一来我们可以得到一个与原计算过程等价的计算过程的集合,通过计算稳定性的验证方法,我们对该集合中的计算过程进行一一验证,直到找到计算稳定的形式,最后我们将这样的稳定计算形式还原为代码的形式形成优化后程序。

为了验证本文提出的基于随机代数变换的数值程序优化方法的有效性,我们完整实现了一个数值程序的自动优化工具,该工具能够以任意精度数值计算程序为输入,将之自动的优化程序执行效率更高的浮点类型的数值程序。同时该工具的可扩展性也非常好,我们工具的转换效果主要取决于工具规则库中规则的丰富程度,而规则库在我们的工具中以模板文件的方式给出,可以非常方便的添加新的规则,从而提升我们优化工具的优化效果。

为了验证优化工具的优化效果,本文主要进行了两个部分的实验。第一个实验的主要目标是iRRAM任意精度数值计算库\cite{10.1007/3-540-45335-0_14}中所自带的经典的11个数值计算程序,我们对这11个任意精度的数值计算程序进行了优化,比较了这些程序优化前后的运行结果以及运行效率。实验表明经过我们优化工具优化得到的优化程序的运行结果与原任意精度程序基本保持了一致,然而程序的执行效率得到了大幅度的提升。第二个实验的实验目标是GLIBC数学函数库中若干个典型的数学函数的实现,我们比较了使用我们工具进行这些函数的开发以及使用普通浮点精度类型进行开发的难易程度以及得到的程序代码的复杂程度,结果表明使用我们的工具可以大大降低数值程序的开发难度,同时也使得程序代码的可读性与可维护性得到了显著提升。

\section{论文结构安排}

本章首先介绍了本文的研究背景,包括数值计算的背景,数值程序在我们日常生活中的重要地位以及保障数值程序正确高效的重要性,并介绍了保障数值程序正确高效的相关工作,包括了程序优化以及程序验证,然后介绍了本文的主要研究工作。本文的后续章节安排如下:

第二章的内容主要介绍本文研究工作的背景知识,主要分为四个部分,包括浮点数相关的背景知识、任意精度计算的背景知识、误差分析常用的两种静态分析方法以及符号执行的相关内容。首先浮点数相关背景知识中,我们主要介绍了浮点数的表示方法以及浮点数计算中产生误差的原因,在任意精度计算相关背景知识中我们主要介绍了任意精度计算的定义,以及需要使用任意精度计算的原因,最后我们介绍了两个常用的任意精度数值计算库。接下来我们介绍了误差分析技术中常用的两种误差分析方法区间算法以及仿射算法,内容主要包括这两种分析方法使用的误差模型并比较了这两种方法的优缺点。最后,在对数值程序的计算程序提取过程中,我们使用到了符号执行技术,因此我们也对其原理进行了简要的介绍。

第三章中我们首先结合一个简单的实例阐释了我们优化方法的基本思想与流程,然后我们对优化方法的优化流程做了一个整体的介绍,包括了稳定性分析、路径提取、随机代数变换以及路径合并这四个不同的步骤,紧接着我们详细介绍了这四个优化流程的具体细节。

第四章的内容主要是我们根据第三章中的优化方法实现了一个数值程序的自动优化工具,简要介绍了该工具的一些实现细节以及使用方法。此外,我们在该工具的基础上设计了两个实验,一个是对比了iRRAM任意精度数值计算库中部分实例程序在我们优化工具优化前后的运行结果以及运行效率,意在说明我们的优化工具能够在不影响程序准确性的情况下大幅提升程序的运行效率。另外一方面我们以GNU科学计算库中典型数学函数为目标,比较了使用我们的优化工具进行数值程序的开发以及使用普通的浮点精度进行数值程序开发的难易程度,结果表明我们的工具可以大大降低开发数值程序的开发难度并且提高程序代码的可读性与可维护性。

第五章中,我们对本文的工作进行了一个简单的总结。首先介绍本文的研究工作以及研究成果,然后给出本文工作中存在的不足以及未来的工作方向和计划。
