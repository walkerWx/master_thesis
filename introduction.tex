% 学位论文的正文应以《绪论》作为第一章
\chapter{绪论}\label{chapter_introduction}

\section{研究背景}

数值计算是一项在理工类各学科各专业中被广泛运用的一项技术。近些年来,随着计算机科学与技术的高速发展,计算机已成为日常工作、生活中不可或缺的工具,在这种情况下,数值计算与计算机的联系变得更为密切。其应用也日益广泛。与计算机科学的其他方向不同,数值计算中的问题和方法有其鲜明的特点,它主要处理连续的物理量,例如时间、距离、温度、电压等等,而不是连续的物理量。同时,数值计算涉及的问题很多都是连续的数学问题,例如求导数、求积分或非线性方程等等,理论上不能通过有限步计算出准确的结果,计算过程往往需要近似。数值计算程序需要通过有限步的迭代得到一个充分接近准确解的近似解。然而,由于计算机不能够准确的表示所有的实数,数值计算程序的计算过程中几乎每一步都存在误差,因此,通过一定的技术手段来保证数值计算程序的正确性相当的重要。

迄今为止,数值计算程序已经被广泛地使用在人们生活和工作中的诸多复杂系统之中,其中包括航空航天、国防、交通运输、核电能源和医疗卫生等诸多安全关键系统。之所以称它们为安全关键系统,是因为它们一旦失效将会导致巨大的经济损失,甚至危害人们的生命安全。然而大多数的数值计算程序都是由浮点类型编写,由浮点类型计算产生的舍入误差而导致的程序缺陷非常的难以检测以及调试。在历史上有与浮点数的误差积累要引起的严重的软件安全事故也是数不胜数,例如1996年6月4日,阿丽亚娜5号运载火箭在发射后37秒被迫自行引爆,究其原因是因为火箭控制系统中控制火箭水平加速度的一个64位浮点数在强制转换为16位整型数值时发生了溢出,导致程式崩溃后处理器发生数字溢流,将感测角度的垂直读值错误的代入到水平值做运算,导致火箭在高速下进行90度水平滚转而崩解,触发自毁装置的启动。类似阿丽亚娜的控制系统,许多的软件系统使用浮点类型程序来编写实数类型算法的代码,这样的算法对于实数类型也许是正确完备的,然而由于浮点数计算误差的存在,按照这样的方式编写的数值计算程序难以保证其正确性。除此之外,由于浮点数舍入误差的存在,一个使用浮点类型实现的数值计算程序的计算结果与正确的任意精度类型实现的计算结果可能有着天壤之别。

\section{研究现状}

既然浮点类型的数值程序存在着如此多的问题,研究人员们也提出了非常多的技术方案来解决这些问题。首先,通常的软件开发人员在开发数值程序时,如果遇到程序的输出与预期的不同的情况,他们会调整其中某些计算过程的代码(例如使用$(x-y)*(x+y)$而不是$x^2-y^2$来计算两个数的平方差),使得最终的结算结果“貌似”正确[]。这样的调试过程往往依赖程序员的开发经验并且很少有规律可循,并且只是使得当前测试的输入的计算结果正确,有可能引入新的计算误差,并且无法覆盖程序的输入域,不具有代表性。

另一种常用的处理浮点类型值程序舍入误差的方式是提高程序精度,例如开发人员可以使用64位双精度浮点类型来代替32位单精度浮点类型,这样做的效果是可以使得数值程序计算结果的误差出现在有效数字的更低位,然而即便是使用任意精度类型(见X.X.X小节),如果软件开发人员没有足够的开发数值程序的经验去选择一个足够的精度,由舍入误差带来的计算结果偏差依然有可能是令人无法忍受的。同时,提高程序的计算精度也会带来额外的计算开销,在一些计算资源受限的系统,例如移动设备上,提高数值程序的精度会产生程序运行率低下的问题。

经验丰富的数值程序开发人员会使用一些数值分析技术来开发一些正确性有保障的数值程序,这样的程序的计算结果能够保证与理想情况下使用实数来计算得到的结果非常的接近。这样的数值分析技术主要包括了前向误差分析(forward error analysis)和后向误差分析(backward error analysis),这样的误差分析技术能够保证程序的计算误差被约束在一个很小的范围内,从而保证程序计算结果的准确性。这样分析技术需要软件开发人员对浮点数的计算过程有着深刻的理解,同时误差分析的过程也非常复杂,难以广泛的应用。

\section{本文工作}

在前文中我们提到,浮点类型的数值程序在计算过程中有舍入操作,所带来的舍入误差会使得程序的正确性难以得到保障,同时浮点类型特有的精度操作也会使得浮点类型数值程序的代码难以阅读以及维护。而如果通过提高精度的方式来提升程序的正确性,又需要软件开发人员具有丰富的数值程序开发经验以提供足够的精度,同时也会随之产生程序计算效率低下,占用较多计算资源的问题。为了解决上述问题,本文提出了一种数值程序的自动优化方法,该方法可以将任意精度类型实现的计算程序自动的转换为高效并且计算稳定的浮点类型的数值计算程序。软件开发人员在开发数值程序时只需要按照在数学上计算该问题的公式以任意精度类型来编写代码,这样的直接按照数学公式来编写的代码是非常清晰的,易于理解也易于维护,紧接着,我们的优化方法可以将这样的代码自动的转换成为高效并且稳定的浮点类型的数值程序。在对程序进行转换的过程中,我们不是对程序中的部分代码片段进行转换,而是将程序的计算过程作为一个整体提取出来,然后对这样一个整体的计算过程进行优化。

在通常情况下,开发一个正确且高效的数值程序需要丰富的数值计算的相关知识,而本文的工作实际上使得软件开发人员在不具备这种专业知识的情况下也能够开发出正确性与效率都得到保障的数值程序。软件开发人员不需要将精力集中在如何处理浮点数的舍入误差,也不必考虑浮点数计算过程中需要避免的一些容易产生较大误差的操作,只需要从数学的角度思考问题,将程序中所有变量的类型看作是实数类型,不必考虑其精度。这样一来,不仅仅大大提升了数值程序的开发效率,还使得开发出来的数值程序非常的符合人们的直觉,也易于维护。

要完成这样的数值程序的自动优化工作,我们首先会对程序的计算过程进行一个整体的提取工作,这个提取过程主要通过符号执行的技术来完成,然后我们寻找需要满足以下两个条件的一个计算过程:
\begin{itemize}
\item 新的计算过程必须与原程序的计算过程在数学意义上是等价的
\item 新的计算过程的浮点精度实现在其输入域上必须是计算稳定的
\end{itemize}
如何寻找满足这样的条件的计算过程是本文工作的主要难点。在本文中,我们提出了一种基于随机代数变换的转换方法来解决这一问题。数值计算专家们定义了一系列数值计算的等价转换规则,我们在原数值程序的计算过程上递归地应用这些等价的转换规则,如此一来我们可以得到一个与原计算过程等价的计算过程的集合,通过计算稳定性的验证方法,我们对该集合中的计算过程进行一一验证,直到找到计算稳定的形式,最后我们将这样的稳定计算形式还原为代码的形式形成优化后程序。

为了验证本文提出的基于随机代数变换的数值程序优化方法的有效性,我们完整实现了一个数值程序的自动优化工具,该工具能够以任意精度数值计算程序为输入,将之自动的优化程序执行效率更高的浮点类型的数值程序。同时该工具的可扩展性也非常好,我们工具的转换效果主要取决于工具规则库中规则的丰富程度,而规则库在我们的工具中以模板文件的方式给出,可以非常方便的添加新的规则,从而提升我们优化工具的优化效果。

为了验证优化工具的优化效果,本文主要进行了两个部分的实验。第一个实验的主要目标是iRRAM任意精度数值计算库中所自带的经典的11个数值计算程序,我们对这11个任意精度的数值计算程序进行了优化,比较了这些程序优化前后的运行结果以及运行效率。实验表明经过我们优化工具优化得到的优化程序的运行结果与原任意精度程序基本保持了一致,然而程序的执行效率得到了大幅度的提升。第二个实验的实验目标是glibc数学函数库中XX个典型的数学函数的实现,我们比较了使用我们工具进行这些函数的开发以及使用普通浮点精度类型进行开发的难易程度以及得到的程序代码的复杂程度,结果表明使用我们的工具可以大大降低数值程序的开发难度,同时也使得程序代码的可读性与可维护性得到了显著提升。

\section{论文结构安排}

本章首先介绍了本文的研究本经,包括数值计算的背景,数值程序在我们日常生活中的重要地位以及保障数值程序正确高效的重要性,并介绍了保障数值程序正确高效的一些常用技术及其不足之处,包括了程序调试,误差分析,提高程序精度以及使用任意精度类型等等,然后介绍了本文的主要研究工作。本文的后续章节安排如下:

第二章的内容主要介绍本文研究工作的背景知识,主要分为四个部分---浮点数相关的背景知识、任意精度计算的背景知识、误差分析常用的两种静态分析方法以及符号执行的相关内容。首先浮点数相关背景知识中,我们主要介绍了浮点数的表示方法以及浮点数计算中产生误差的原因,在任意精度计算相关背景知识中我们主要介绍了任意精度计算的定义,以及需要使用任意精度计算的原因,最后我们介绍了两个常用的任意精度数值计算库。接下来我们介绍了误差分析技术中常用的两种误差分析方法区间算法以及仿射算法,内容主要包括这两种分析方法使用的误差模型并比较了这两种方法的优缺点。最后,在对数值程序的计算程序提取过程中,我们使用到了符号执行技术,因此我们也对其原理进行了简要的介绍。

第三章中我们对我们优化方法的优化流程做了一个整体的介绍,包括了稳定性分析、路径提取、随机代数变换以及路径合并这四个不同的步骤,然后对于整个优化流程中每一个具体的模块我们进行了详细的介绍。最后,我们结合一个程序优化实例具体阐释了我们优化方法的优化流程。

第四章的内容主要是我们根据第三章中的优化方法实现了一个数值程序的自动优化工具,简要介绍了该工具的一些实现细节以及使用方法。此外,我们在该工具的基础上设计了两个实验,一个是对比了iRRAM任意精度数值计算库中部分实例程序在我们优化工具优化前后的运行结果以及运行效率,意在说明我们的优化工具能够在不影响程序准确性的情况下大幅提升程序的运行效率。另外一方面我们以glibc的数学函数库中典型数学函数为目标,比较了使用我们的优化工具进行数值程序的开发以及使用普通的浮点精度进行数值程序开发的难易程度,结果表明我们的工具可以大大降低开发数值程序的开发难度并且提高程序代码的可读性与可维护性。

第六章总,我们对本文的工作进行了一个简单的总结。首先介绍本文的研究工作以及研究成果,然后给出本文工作中存在的不足以及未来的工作方向和计划。
