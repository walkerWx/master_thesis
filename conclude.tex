\chapter{论文总结}\label{chapter_concludes}

\section{论文主要工作}

本文所关注的主要问题是数值计算程序的优化问题,优化目标是在保障数值计算程序正确性的前提下提高数值计算程序的运行效率。绝大多数的数值计算程序均使用浮点精度类型实现,但是浮点精度类型无法精确的表示所有的实数,所以浮点精度类型的计算操作存在舍入误差,并且在计算过程中的容易产生误差积累的问题,由于上述浮点精度类型固有的问题导致了开发浮点精度类型的数值计算程序需要具备相当的领域知识,并且开发效率低下,程序易出错。另一方面,科学家们提出的任意精度计算的方法虽然一定程度上解决了数值计算程序正确性的问题,但是使用任意精度计算存在着计算效率低下的问题。

针对以上提出的数值计算程序在开发中存在的问题,本文主要做了三个方面的工作,具体包括:

\begin{itemize}

    \item 本文提出了一种数值计算程序的自动优化方法,可以将任意精度的数值计算程序自动优化成为与之计算等价的正确并且高效的浮点精度的数值计算程序。通过这种优化方法,数值计算程序的开发人员可以直接编写清晰易读易维护的任意精度的数值计算程序,不必考虑浮点精度程序的误差处理的种种细节,降低了数值计算程序的开发难度同时也保障了数值计算程序的正确性。为了完成这一自动优化过程,本文创新性的提出了一种基于随机代数变换的等价计算过程搜索算法,通过将数值计算专家们给出的等价转化规则形成规则库并在原计算过程上迭代的运用等价转化规则进行转换的形式,我们可以找到一个计算不稳定的计算过程的等价的稳定计算形式。

    \item 基于上述优化方法,我们完整实现了一个数值计算程序的自动优化工具,该工具总共分为四个不同的模块,包括路稳定性分析模块,路径提取模块,随机代数变换模块以及路径合并模块,该工具以任意精度实现的数值计算程序为输入,稳定性分析模块可以分析得到原程序的稳定输入域,不稳定输入域以及未知输入域,路径提取模块可以对原程序的计算过程进行一个整体的提取工作,随机代数变换模块基于规则库可以寻找到一个不稳定计算过程的计算稳定形式,路径合并模块将不同计算路径下的计算过程组合成为最终优化后的程序。

    \item 为验证工具在真实程序上的优化效果以及使用该工具进行数值程序开发效率的提升,我们使用该优化工具进行了两部分实验,实验一我们选取了任意精度数值计算库iRRAM中的11个典型计算程序对其进行了优化,比较了优化前后的程序输出,相对误差,程序运行时间等指标,我们发现我们的工具可以在保障数值计算正确性的前提下极大地提高程序的运行效率。另一部分实验我们选取了GNU科学计算库中的部分数学函数,分别使用任意精度、浮点精度进行开发,比较两种代码的各种软件度量指标,包括程序容量,圈复杂度以及可维护性指数,我们发现使用我们工具进行数值计算开发可以显著提升程序开发效率,并且程序代码的复杂性相比于浮点精度得到了极大的降低。

\end{itemize}

\section{未来工作}

本文工作还可以进一步完善,未来的工作将主要集中在以下几个方面:

\begin{itemize}
    \item 当前方法中在生成等价计算过程时,采用的是完全随机的策略,即应用的等价转换规则是随机选取的,进行应用规则进行转换的计算过程也是在等价表达式集合中随机选取的,这样可能需要迭代多次才能够寻找到一个稳定的计算过程,寻找等价的稳定计算形式的过程耗时较长,后续我们希望在运用规则进行等价转换寻找稳定形式的策略上做进一步的改进,运用一些例如启发式方法的方式来进行搜索,提高搜索算法的效率。
    \item 另外当前随机代数变换的规则库规模较小,只有几十条等价转换规则,而我们工具的优化能力与这个规则库的完善程度是密切相关的,后续希望进一步完善规则库中的规则,以提高我们优化工具的优化能力。
\end{itemize}